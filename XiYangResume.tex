% (c) 2002 Matthew Boedicker <mboedick@mboedick.org> (original author) http://mboedick.org
% (c) 2003-2007 David J. Grant <davidgrant-at-gmail.com> http://www.davidgrant.ca
% (c) 2008 Nathaniel Johnston <nathaniel@nathanieljohnston.com> http://www.nathanieljohnston.com
% (c) 2011 Scott Clark <sc932@cornell.edu> http://cam.cornell.edu/~sc932
%

%This work is licensed under the Creative Commons Attribution-Noncommercial-Share Alike 2.5 License. To view a copy of this license, visit http://creativecommons.org/licenses/by-nc-sa/2.5/ or send a letter to Creative Commons, 543 Howard Street, 5th Floor, San Francisco, California, 94105, USA.

\documentclass[letterpaper,11pt]{article}
\newlength{\outerbordwidth}
\pagestyle{empty}
\raggedbottom
\raggedright
\usepackage[svgnames]{xcolor}
\usepackage{framed}
\usepackage{tocloft}
\usepackage{etoolbox}
\usepackage{hyperref}
\robustify\cftdotfill

%-----------------------------------------------------------
%Edit these values as you see fit
\setlength{\outerbordwidth}{3pt}  % Width of border outside of title bars
\definecolor{shadecolor}{gray}{0.75}  % Outer background color of title bars (0 = black, 1 = white)
\definecolor{shadecolorB}{gray}{0.93}  % Inner background color of title bars

%-----------------------------------------------------------
%Margin setup
\setlength{\evensidemargin}{-0.25in}
\setlength{\headheight}{-0.25in}
\setlength{\headsep}{0in}
\setlength{\oddsidemargin}{-0.25in}
\setlength{\paperheight}{11in}
\setlength{\paperwidth}{8.5in}
\setlength{\tabcolsep}{0in}
\setlength{\textheight}{9.75in}
\setlength{\textwidth}{7in}
\setlength{\topmargin}{-0.3in}
\setlength{\topskip}{0in}
\setlength{\voffset}{0.1in}

%-----------------------------------------------------------
%Custom commands
\newcommand{\resitem}[1]{\item #1 \vspace{-2pt}}
\newcommand{\resheading}[1]{\vspace{8pt}
  \parbox{\textwidth}{\setlength{\FrameSep}{\outerbordwidth}
    \begin{shaded}

\setlength{\fboxsep}{0pt}\framebox[\textwidth][l]{\setlength{\fboxsep}{4pt}\fcolorbox{shadecolorB}{shadecolorB}{\textbf{\sffamily{\mbox{~}\makebox[6.762in][l]{\large #1} \vphantom{p\^{E}}}}}}
    \end{shaded}
  }\vspace{-5pt}
}

\newcommand{\ressubheading}[4]{
\begin{tabular*}{6.5in}{l@{\cftdotfill{\cftsecdotsep}\extracolsep{\fill}}r}
		\textbf{#1} & #2 \\
		\textit{#3} & \textit{#4} \\
\end{tabular*}\vspace{-6pt}}

\newcommand{\ressubheadingTwo}[2]{
\begin{tabular*}{6.5in}{l@{\cftdotfill{\cftsecdotsep}\extracolsep{\fill}}r}
		\textbf{#1} & #2
\end{tabular*}\vspace{-6pt}}

%-----------------------------------------------------------
\begin{document}

\begin{tabular*}{7in}{l@{\extracolsep{\fill}}r}

\textbf{\Large Xi Yang} & \textbf{\today} \\
\texttt{hiyangxi@gmail.com} & \texttt{\href{https://yangxi.github.io}{yangxi.github.io}} \\
\texttt{Software Engineer at Confluent}

\end{tabular*}

%%%%%%%%%%%%%%%%%%%%%%%%%%%%%%

\resheading{Goal}

Craft platforms with which humane people can control the fire of computing.

\resheading{Research Interests}
%%%%%%%%%%%%%%%%%%%%%%%%%%%%%%


My research focuses on creating new tools for understanding
and controlling the run-time behavior of complex computer systems.


%%%%%%%%%%%%%%%%%%%%%%%%%%%%%%

\resheading{Education}

%%%%%%%%%%%%%%%%%%%%%%%%%%%%%%

\begin{itemize}

\item
	\ressubheading{Australian National University}{Canberra,
          Australia}{Ph.D. Computer Science}{Oct. 2011 - May. 2019}
	\begin{itemize}
                \resitem{Thesis: SHIM and Its Applications}
		\resitem{Supervisors: Prof. Steve Blackburn, Prof. Kathryn McKinley}
	\end{itemize}

\item
	\ressubheading{Australian National University}{Canberra, Australia}{Mphil
          Computer Science }{May 2009 - March 2011}
	\begin{itemize}
		\resitem{Thesis: Locality Aware Zeroing: Exploiting Both
                    Hardware and Software Semantics}
                \resitem{Supervisors: Prof. Steve Blackburn, Prof. Kathryn McKinley}
	\end{itemize}

\item
	\ressubheading{University of Electronic Science and Technology of
          China}{Chengdu, China}{B.S. Computer Science}{Sept. 2004 - July 2008}
        \begin{itemize}         
          \resitem{Thesis: RTEMS on L4 Microkernel}
          \resitem{Supervisor: Prof. Kevin Elphinstone}
          \resitem{I did the thesis at the ERTOS group (NICTA), University of New
            South Wales as a visiting student.}
          \end{itemize}
\end{itemize}

%%%%%%%%%%%%%%%%%%%%%%%%%%%%%%

\resheading{Honors and Awards}

%%%%%%%%%%%%%%%%%%%%%%%%%%%%%%
\begin{itemize}
\item{Paper recognized as Honorable Mention in {\bf IEEE Micro Top Picks from the 2015 Computer
      Architecture Conferences}. {\it ``Computer Performance Microscopy with SHIM''}}
\item{{\bf 2012 Google Australia PhD Fellowship in Energy Aware Computing:} One year fellowship, one of three
  awarded in Australia and 40 worldwide.}
\item{Paper selected for {\bf Communications of the ACM Research
      Highlights}. {\it ``Looking
  Back and Looking Forward: Power, Performance, and Upheaval''}}
\item{Paper selected for {\bf IEEE Micro Top Picks from the 2011 Computer
      Architecture Conferences}. {\it ``What Is Happening to Power,
      Performance, and Software?''}}
\end{itemize}

%%%%%%%%%%%%%%%%%%%%%%%%%%%%%%

\resheading{Publications}

%%%%%%%%%%%%%%%%%%%%%%%%%%%%%%

\begin{itemize}

\item {\bf X. Yang}, S. M. Blackburn, and K. S. McKinley, \href{https://github.com/yangxi/papers/raw/master/elfen-atc-2016.pdf} {\color{blue} \bf
    "Elfen Scheduling: Fine-Grain Principled Borrowing from Latency-Critical
    Workloads using Simultaneous Multithreading"}, in Proceedings of the 2016
  USENIX Annual Technical Conference {\bf (USENIX ATC) }, Denver, CO, June 22-24, 2016.

\item {\bf X. Yang}, S. M. Blackburn, and K. S. McKinley, \href{https://github.com/yangxi/papers/raw/master/shim-isca-2015.pdf} {\color{blue} \bf
    "Computer Performance Microscopy with SHIM"}, in Proceedings of the 42nd International 
     Symposium on Computer Architecture {\bf (ISCA) }, Portland,
     OR, June 13-17, 2015.

\item R. Shahriyar, S. M. Blackburn, {\bf X. Yang}, and K. M. McKinley,
  \href{https://github.com/yangxi/papers/raw/master/rcix-oopsla-2013.pdf}{\color{blue} \bf
      ``Taking Off the Gloves with Reference Counting Immix''}, in Proceedings of 
      the 2013 ACM SIGPLAN Conference on Object-Oriented Programming, Systems,
      Languages, and Applications {\bf (OOPSLA)}, Indianapolis, IN,
      October 26-31, 2013.

\item{\bf X. Yang}, D. Frampton, S. M. Blackburn, and A. L. Hosking,
  \href{https://github.com/yangxi/papers/raw/master/barrier-ismm-2012.pdf}{\color{blue} \bf
    "Barriers Reconsidered, Friendlier Still!"}, in Proceedings of the 2012
  International Symposium on Memory Management {\bf (ISMM) }, Beijing,
  China, June 15-16, 2012.

\item H. Esmaeilzadeh, T. Cao, {\bf X. Yang}, S. M. Blackburn, and
    K. S. McKinley,
    \href{https://github.com/yangxi/papers/raw/master/powerperf-micro-2012.pdf}{\color{blue}
      \bf "What is Happening to Power, Performance, and
      Software?,"}, {\bf IEEE Micro}, vol. 32, pp. 110-121, 2012. 

\item H. Esmaeilzadeh, T. Cao, {\bf X. Yang}, S. M. Blackburn, and
    K. S. McKinley, \href{https://github.com/yangxi/papers/raw/master/powerperf-cacm-2012.pdf}{\color{blue}\bf "Looking Back and Looking Forward: Power, Performance,
      and Upheaval,"}, {\bf Communications of the ACM}, vol. 55, iss. 7, pp. 105-114, 2012.

\item {\bf X. Yang}, S. M. Blackburn, D. Frampton, J. B. Sartor, and
  K. S. McKinley,
  \href{https://github.com/yangxi/papers/raw/master/zero-oopsla-2011.pdf}
  {\color{blue} \bf ``Why Nothing Matters: The Impact of Zeroing''}, in Proceedings of 
the 2011 ACM SIGPLAN Conference on Object-Oriented Programming, Systems, Languages, and Applications {\bf (OOPSLA)}, Portland, OR, October 22-27, 2011.

\item H. Esmaeilzadeh, S. M. Blackburn, T. Cao, {\bf X. Yang}, and
  K. S. McKinley,
  \href{https://github.com/yangxi/papers/raw/master/powerperf-asplos-2011.pdf}
  {\color{blue} \bf ``Looking Back on the Language and Hardware Revolution:
    Measured Power, Performance, and Scaling''}, in Proceedings of the 16th
  International Conference on Architectural Support for Programming Languages
  and Operating Systems {\bf (ASPLOS)}, Newport Beach, CA, USA, March 5-11,
  2011.
\end{itemize}

%%%%%%%%%%%%%%%%%%%%%%%%%%%%%%

\resheading{Research Experience}

%%%%%%%%%%%%%%%%%%%%%%%%%%%%%%


\begin{itemize}

\item
	\ressubheading{Australian National University}{Canberra,
          Australia}{Ph.D. Computer Science}{Oct. 2011 - May. 2019}
	\begin{itemize}
                \resitem {Designed and implemented Tailor, a real-time
                  latency controller that uses a SHIM-based
                  high-frequency profiler and an application-level
                  network proxy to continuously monitor and act on
                  hazardous system behaviors.}
                  
                \resitem {Designed and implemented Elfen {\texttt{\color{blue} https://github.com/yangxi/elfen}}, a scheduler
                  that co-runs latency-critical workloads and batch
                  workloads on the same SMT CPUs. Elfen significantly
                  improves the datacenter CPU utilization without
                  affecting latency-critical workloads.
                   {\bf USENIX ATC 2016}}

                \resitem {Designed and implemented SHIM {\texttt{\color{blue} https://github.com/yangxi/SHIM}}, a continuous profiler
                that samples at resolutions as fine as 15 cycles; three to five
                orders of magnitude finer than current continuous
                profilers. {\bf ISCA 2015}}

		\resitem {Evaluated the overhead of managed language runtime barriers. {\bf ISMM 2012}}

	\end{itemize}

\item
	\ressubheading{Australian National University}{Canberra, Australia}{Mphil
          Computer Science }{May 2009 - March 2011}
	\begin{itemize}
          \resitem {Analyzed the overhead of zero initialization,
            discovered that the
          cost is high for Java workloads, and proposed three better designs to reduce
          the direct and indirect zeroing costs simultaneously. The three new
          zeroing approaches are in JikesRVM now.  {\bf OOPSLA 2011}}

          \resitem {Analyzed directly measured power and performance
            of five generations of Intel CPUs executing 61 diverse benchmarks with a
          rigorous methodology. {\bf ASPLOS 2011}}
          
	\end{itemize}

\item
	\ressubheading{University of Electronic Science and Technology of
          China}{Chengdu, China}{B.S. Computer Science}{Sept. 2004 - July 2008}
        \begin{itemize}         
          \resitem {Ported the RTEMS, a single address space RTOS, to the L4
          Microkernel, and evaluated the performance. Contributed parts of the project,
          PXA255 and Gumstix BSP, to the open source RTEMS operating system.}
      \end{itemize}
    \end{itemize}


%%%%%%%%%%%%%%%%%%%%%%%%%%%%%%

\resheading{Working Experience}

%%%%%%%%%%%%%%%%%%%%%%%%%%%%%%


\begin{itemize}

  \item
	\ressubheading{Confluent}{Sydney, Australia}{Software Engineer}{2018 - Date}
	\begin{itemize}
                \resitem {Build a distributed performance evaluation
                  framework for the Apache Kafka platform.}
        \end{itemize}
  \item
	\ressubheading{Terrain Data}{Sydney, Australia}{Software Engineer}{2017 - 2018}
	\begin{itemize}
                \resitem {Helped to create the first interactive algorithm
                  management system with which non-tech users can
                  manage, annotate, and search their structural data.}
         \end{itemize}
\end{itemize}

   
%%%%%%%%%%%%%%%%%%%%%%%%%%%%%%

\resheading{Community Services}


%%%%%%%%%%%%%%%%%%%%%%%%%%%%%%
\begin{itemize}
  \item
  \ressubheadingTwo{Program Committee Member}{}
  \begin{itemize}
  \resitem{OOPSLA'13 AEC}
  \resitem{PLDI'16 ERC}
  \resitem{OOPSLA'16 AEC}
  \end{itemize}
\end{itemize}

\end{document}
